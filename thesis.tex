% Root File for a UC Dissertation / Thesis
% UCD thesis class: c/o Shwaine <shwaine@shwaine.com>
%
% modified by Dylan Beaudette, 2006,2010
% modified  and source uploaded to github by Alex Mandel, 2014
% Source code available at http://github.com/wildintellect/ucdthesis
\documentclass[11pt]{ucdthesis}
%\documentclass[10pt,twoside,final]{ucdthesis}
% \documentclass[10pt,twoside,draft]{ucdthesis}
% \documentclass[10pt,oneside,final]{ucdthesis}

% TODO: this makes strange things happen in the header...
% this is the version that grad studies wants
% \documentclass[11pt,oneside,final]{ucdthesis}

% when we are giving people drafts, use more of the page:
% \usepackage[letterpaper,left=0.75in,right=0.75in,top=1.25in,bottom=0.5in]{geometry}
% need this for the \foreach command
\usepackage{tikz}

% Turn on single spacing with \ssp.
% Turn on double spacing with \dsp.
% By default, your dissertation is double spaced, as is required by UCD.

% spacing in figures and tables and their captions can be
% changed here (\ssp for single-space, empty for same as surrounding
% text); for this to work, the command \figsp has to be included
% in every figure and table right after the \begin{figure}
% \def\figsp{\ssp}
%\def\figsp{}


% useful for drafts
% line numbers:
% http://www.ctan.org/tex-archive/help/Catalogue/entries/lineno.html
%
% \usepackage{lineno}

%
% SVN integration
%
% \usepackage{svnkw}

% customized headers
%
% http://www.ctan.org/tex-archive/help/Catalogue/entries/fancyhdr.html
% \usepackage{fancyhdr}

% a better verbatim environment: c/o Pete Dirac
% use like this:
%   \begin{Verbatim}[fontsize=8]
%       foobar
%    \end{Verbatim}
% \usepackage{fancyvrb}


% more flexible math support
\usepackage{amsmath}

% allow some pages to be landscape
\usepackage{lscape}

% more flexible definition of table environments
\usepackage{ctable}

%need this for \includegraphics{}
\usepackage{graphicx}

%enable the listings package specifically for including programming code
\usepackage{listings}

%Special hack to make code listings not break pages, fyi they must be short then
\usepackage{float}
\floatstyle{plain} % optionally change the style of the new float
\newfloat{Code}{H}{myc}

%test alternative to listings package minted which requires the python pygments package
%minted was installed to latex by hand
%\usepackage{minted}

% TODO: use the new subfig package instead
% http://www.ctan.org/tex-archive/macros/latex/contrib/subfig/
%
%use this to put figures side by side
\usepackage{subfigure}

 % PDF links --- > breaks with some bibliography entries
% \usepackage{hyperref}

% nice looking, parenthetical references
\usepackage[sorting=nyt,natbib=true,citestyle=authoryear,bibstyle=authoryear,maxnames=3,refsection
=chapter]{biblatex}
%\usepackage{chapterbib} % incompatible with biblatex
\bibliography{proposal}
\defbibheading{bibliography}{%
	\section{References}
	}
	
% bibliography can be single-spaced for UC thesis format
\appto{\bibsetup}{\ssp}
	

%make the index
% \usepackage{makeidx}
% \makeindex

% custom colors
\usepackage{color}
% make a color for comments
\definecolor{MyDarkBlue}{rgb}{0,0.08,0.45}

% customized captions with bold label and small, italic text
% table captions are located above tables
% http://www.kronto.org/thesis/tips/custom-captions.html
% http://www.ctan.org/tex-archive/macros/latex/contrib/caption/
% does this have any effect?
\usepackage{caption}
%\renewcommand{\captionfont}{\small\itshape}

%
\usepackage[hypcap,font=singlespacing]{caption}
\usepackage{subcaption}
% modern method for setting up captions\
\captionsetup{margin=10pt,font=footnotesize,labelfont=bf}
%
% fix so that table captions have correct spacing
\captionsetup[table]{position=top}



% %
% %  fit more material on the page:
% %
%
% reset some float-controlling parameters
\renewcommand{\floatpagefraction}{0.8}	% require fuller float pages

% N.B.: floatpagefraction MUST be less than topfraction !!
\renewcommand{\topfraction}{0.9}	% max fraction of floats at top
\renewcommand{\bottomfraction}{0.8}	% max fraction of floats at bottom


% PDF formatting options, indexing, hyperlinking, with control over link style
%Set PDF Metadata
\usepackage[pdftex,
            pdfauthor={Author Name},
            pdftitle={Title Here},
            pdfsubject={Subject},
            pdfkeywords={Comma, List, Keywords},
            %pdfproducer={Latex with hyperref, or other system},
            %pdfcreator={pdflatex, or other tool}
            ]{hyperref}
\hypersetup{
	%driver=pdftex,
	colorlinks=true,
	urlcolor=blue,
	linkcolor=blue,          % color of internal links
    citecolor=blue,        % color of links to bibliography
    filecolor=magenta
}

%Use an additional package to make bookmarks point to the top to tables, figures and listings
\usepackage[all]{hypcap}

%Alex's customizations
\usepackage{indentfirst} %Indents first paragraph of chapter
\usepackage{datatool} %Allows import of csv and other data-tables
\usepackage{varwidth}
\usepackage{color}
\usepackage{rotating}
\AtEveryBibitem{\clearfield{month}} %Cleaner references without month being printed

%%% Document Portion:
\begin{document}


%
%% Title, Front Matter, and Abstract:
\input{title_and_frontmatter}

%
% the chapters
%

% set page style:
% make the chapter and section smaller, chapter and section numbers are removed
% fancyplain will keep the page numbers at the bottom of all pages
%\pagestyle{fancyplain} %Note the \fancyplain command !!!
%\renewcommand{\chaptermark}[1]{\markboth{\small{#1}}{}}
%\renewcommand{\sectionmark}[1]{\markright{\small{#1}}{}}


% TODO: this is only for draft copies !!
% start line number printing
%
% \linenumbers



% chapter 1
\chapter{Real Title Here}
\input{chapter1} %This looks for chapter1.tex

% chapter 2
\chapter{Real Title Here}
\chapter{Structure, variation, and assembly of the root-associated microbiomes of rice}

Joseph Edwards\footnote[1]{Department of Plant Biology, University of California, Davis}, Cameron Johnson$^1$, Christian Santos$^1$, Eugene Lurie$^1$, Natraj Kumar Podishetty$^1$, Srijak Bhatnagar\footnote[1]{Department Medical Microbiology and Immunology, University of California, Davis} Jonathan A. Eisen$^2$ and Venkatesan Sundaresan$^1$

\section{Abstract}
Plants depend upon beneficial interactions between roots and microbes for nutrient availability, growth promotion, and disease suppression. High-throughput sequencing approaches have provided recent insights into root microbiomes, but our current understanding is still limited relative to animal microbiomes. Here we present a detailed characterization of the root-associated microbiomes of the crop plant rice by deep sequencing, using plants grown under controlled conditions as well as field cultivation at multiple sites. The spatial resolution of the study distinguished three root-associated compartments, the endosphere (root interior), rhizoplane (root surface), and rhizosphere (soil close to the root surface), each of which was found to harbor a distinct microbiome. Under controlled greenhouse conditions, microbiome composition varied with soil source and genotype. In field conditions, geographical location and cultivation practice, namely organic vs. conventional, were factors contributing to microbiome variation. Rice cultivation is a major source of global methane emissions, and methanogenic archaea could be detected in all spatial compartments of field-grown rice. The depth and scale of this study were used to build coabundance networks that revealed potential microbial consortia, some of which were involved in methane cycling. Dynamic changes observed during microbiome acquisition, as well as steady-state compositions of spatial compartments, support a multistep model for root microbiome assembly from soil wherein the rhizoplane plays a selective gating role. Similarities in the distribution of phyla in the root microbiomes of rice and other plants suggest that conclusions derived from this study might be generally applicable to land plants. %This looks for chapter2.tex



% note that the 'plainnat' style does not allow URL's in the bibtex entry
%
% some ideas here:
% http://bib2web.djvuzone.org/bibtex.html
%

% reset the page style
%\pagestyle{plain}


% To enable this it will need to be added to toc so it's not in a chapter
%\printbibliography[heading=bibliography]

% the appendix:
% there are several sections, that don't really fit into the main chapters
%
\part*{\addcontentsline{toc}{part}{Appendices}Appendices}
\appendix

% reset page style to fancy
%\pagestyle{fancyplain}

 \input{appendix}


\end{document}
