\chapter{Structure, variation, and assembly of the root-associated microbiomes of rice}

Joseph Edwards\footnote[1]{Department of Plant Biology, University of California, Davis}, Cameron Johnson$^1$, Christian Santos$^1$, Eugene Lurie$^1$, Natraj Podishetty$^1$, Srijak Bhatnagar\footnote[2]{Department Medical Microbiology and Immunology, University of California, Davis}, Jonathan A. Eisen$^2$ and Venkatesan Sundaresan$^1$

\section{Abstract}
Plants depend upon beneficial interactions between roots and microbes for nutrient availability, growth promotion, and disease suppression. High-throughput sequencing approaches have provided recent insights into root microbiomes, but our current understanding is still limited relative to animal microbiomes. Here we present a detailed characterization of the root-associated microbiomes of the crop plant rice by deep sequencing, using plants grown under controlled conditions as well as field cultivation at multiple sites. The spatial resolution of the study distinguished three root-associated compartments, the endosphere (root interior), rhizoplane (root surface), and rhizosphere (soil close to the root surface), each of which was found to harbor a distinct microbiome. Under controlled greenhouse conditions, microbiome composition varied with soil source and genotype. In field conditions, geographical location and cultivation practice, namely organic vs. conventional, were factors contributing to microbiome variation. Rice cultivation is a major source of global methane emissions, and methanogenic archaea could be detected in all spatial compartments of field-grown rice. The depth and scale of this study were used to build coabundance networks that revealed potential microbial consortia, some of which were involved in methane cycling. Dynamic changes observed during microbiome acquisition, as well as steady-state compositions of spatial compartments, support a multistep model for root microbiome assembly from soil wherein the rhizoplane plays a selective gating role. Similarities in the distribution of phyla in the root microbiomes of rice and other plants suggest that conclusions derived from this study might be generally applicable to land plants.\footnote[3]{All supplementary material and datasets can be found with the online version of this article: \url{http://www.pnas.org/content/112/8/E911}.}